\documentclass[8pt]{article}
\usepackage[utf8]{inputenc}
\usepackage[a4paper, landscape, total={11in, 7in}]{geometry}
\usepackage{rotating}
\usepackage{colortbl}
\usepackage{graphicx}
\renewcommand{\arraystretch}{.5}

\begin{document}

\section{Supplementary Note 1.}
A clonotype designates a collection of {\alpha} T or B cells that bear the same adaptive immune receptors, and thus can be regarded as descendants of a common antecedent cell, recognizing the same epitope. Applying this definition to single-cell TCR sequencing datasets would mean that cells sharing identical sequences of both alpha and beta TCRs make up a clonotype. Contrary to what would be expected based on the previously described mechanism of allelic exclusion, single-cell sequencing datasets feature a considerable number of cells with more than one adaptive immune receptor sequences. \par
Since their existence violates current canon (Brady et al. 2010), most TCR analysis tools ignore cells with more than one sequence (Fischer et al. ; Zhang et al. 2018) or take the sequence with the highest expression level as the only valid choice (Afik et al. 2017). While in some cases these cells might indeed represent artifacts (e.g. doublets of a CD8+ and a CD4+ T cell engaged in an immunological synapse), there is an increasing amount of evidence in support of a dual TCR population (Schuldt and Binstadt 2019; Ji et al. 2010). Given their abundance in empirical data, we are convinced that instead of ignorance, an analysis tool should at least offer the choice of including this elusive cell type. \par
Scirpy attempts to address this problem by introducing a T cell model (similar to the one suggested by clonotype networks in TraCeR (Stubbington et al. 2016)), where T cells are allowed to have a primary and a secondary pair of alpha and beta chains. The primary pair consists of the alpha chain with the highest read count and the beta chain with the highest read count. Likewise, the secondary pair is the pair of chains with the second-highest expression level. If more than two variants of a chain are recovered for the same cell, those are ignored based on the assumption that each cell has only two copies of the chromosome set. These cells are also flagged as 'multichain' cells and can later be discarded from downstream analysis. The user can also choose if secondary TCRs shall be included in the analysis or not.
Another decision when implementing a clonotype definition is whether it should be based on the nucleotide or amino acid sequence. Although nucleotide sequence reflects common origin, epitope reactivity is determined by the amino acid sequence. This has made the amino acid sequence of the CDR3 region a common choice for clonotype definition by existing tools already and Scirpy conforms to this consensus. Given both the nucleotide and amino acid sequences of the CDR3 regions, it also possible to analyse the probability of convergent evolution of TCR sequences toward a given amino acid sequence as a result of selection pressure by epitopes. \par
Clonotypes, as well as individual cells can be grouped together based on shared CDR3 amino acid sequences (despite differences at the DNA level), shared primary or secondary chain pairs, or (as carried out by TraCeR (Stubbington et al. 2016)), based on shared alpha or shared beta chains. The biological relevance of such networks would be the clustering of cells that recognize the same epitope. Inspired by recent papers stating the similar TCR sequences also share epitope targets (Glanville et al. 2017; Dash et al. 2017; Fischer et al. ), we aimed at implementing a more general approach, where different layers of TCR sequence similarity can be integrated into a global, epitope-focused cell similarity network. The possibility of linking similar clonotypes together also provides an opportunity to correct for information loss due to sequencing depth, possibly resulting in recovering of only an alpha or a beta chain for some cells. \par

\end{document}
